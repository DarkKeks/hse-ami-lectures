\documentclass[a4paper]{article}

\usepackage[T2A]{fontenc}
\usepackage[utf8]{inputenc}
\usepackage[english, russian]{babel}

\usepackage[top=0.8in,bottom=0.8in,left=0.8in,right=0.8in,headheight=110pt]{geometry}
\usepackage{mathtools, amsthm, amssymb}
\usepackage{indentfirst}
\usepackage{enumitem}
\usepackage[unicode=true, colorlinks=true, linkcolor=blue]{hyperref}

\renewcommand\qedsymbol{$\blacktriangle$}

\begin{document}
    \section*{ДЗ к Семинар 1}

    \begin{enumerate}
    \item
        Используя метод математической индукции доказать следующее
        \begin{enumerate}[label=\alph*)]
        \item 
            $6 | n(2n^2 - 3n + 1)$ для любого $n \in \mathbb{N}$ 

            запись $a|b$ читается ``$a$ делит (нацело) $b$''
        \item
            $\sum_{k=1}^n k^3 = 1^3 + 2^3 + \dots + (n - 1)^3 + n^3 = \left( \frac{n(n + 1)}{2} \right)^2$

            обратите внимание, что $\sum_{k=1}^n k^3 = \left( \sum_{k=1}^n k \right)^2$
        \item 
            $\sqrt{n} < 1 + \frac{1}{\sqrt{2}} + \dots + \frac{1}{\sqrt{n}} < 2 \sqrt{n}$ для любого $n \in \mathbb{N}$

            hint: $\sqrt{a} - \sqrt{b} = \frac{a - b}{\sqrt{a} - \sqrt{b}}$
        \end{enumerate}

    \item 
        Разобраться в следующем доказательстве
        
        прежде чем читать доказательство, можно попробовать доказать самостоятельно, но это не просто

        Пусть $a_1, a_2, \dots, a_n$ -- произвольные положительные числа.
        Тогда, для любого $n \in \mathbb{N}$:
        \begin{equation}
            \frac{a_1 + \dots + a_n}{n} \geq \sqrt[n]{a_1 \cdot a_2 \cdots a_n}
            \tag{$\star$}
        \end{equation}

        т.е. среднее арифметическое не меньше среднего геометрического

        \begin{proof}
            $\sqrt{a_1 \cdot a_2} \leq \frac{a_1 + a_2}{2}$ // действительно, $(\sqrt{a_1} - \sqrt{a_2})^2 \geq 0$ //
            
            применим этот результат дважды:
            \[
                \sqrt[4]{a_1 a_2 a_3 a_4} 
                \leq \frac{\sqrt{a_1 a_2} + \sqrt{a_3 a_4}}{2} 
                \leq \frac{\frac{a_1 + a_2}{2} + \frac{a_3 + a_4}{2}}{2} 
                = \frac{a_1 + a_2 + a_3 + a_4}{4}.
            \]

            Пользуясь индукцией мы получаем, что
            \begin{equation}
               (a_1 a_2 \dots a_{2^k})^{\frac{1}{2^k}} \leq \frac{a_1 + \dots + a_{2^k}}{2^k} \text{ для } \forall k
               \tag{$\star\star$}
           \end{equation}

           [Проверьте эту индукцию аккуратно!]

           Заметим, что мы доказали ($\star$) не для всех $n \in \mathbb{N}$, а только для степеней двойки (т.е. $P_1, P_2, P_4, P_8, \dots$)

           А как доказать для остальных $n \in \mathbb{N}$?

           На самом деле все уже доказано, просто надо понять почему.

           Пусть $n \in \mathbb{N}$ -- произвольно и $a_1, a_2, \dots, a_n$ наши числа $\implies$ существует такое $k \in \mathbb{N}$, что $n \leq 2^k$.

           Положим $b_j = \begin{cases}
               a_i, & \text{если } i \leq n \\
               A, & \text{если } n < i \leq 2^k
           \end{cases}$, где $A = \frac{a_1 + \dots + a_n}{n}$

           Применим ($\star\star$) к $b_1, b_2, \dots, b_{2^k} \implies \left(a_1 a_2 \dots a_n A^{2^k - n}\right)^\frac{1}{2^k} \leq \frac{\overbrace{a_1 + \dots + a_n}^{= nA} + (2^k - n) A}{2^k} = A \\ \implies (a_1 \dots a_n)^\frac{1}{2^k} \leq A^\frac{n}{2^k} \implies (\star)$
        \end{proof}

    \item
        Полезно знать следующий трюк

        Как найти сумму 
        \[
            S = 1 + 2 \cdot \frac{1}{3} + 3 \cdot \left(\frac{1}{3}\right)^2 + \dots + n \left(\frac{1}{3}\right)^{n - 1} = \text{ ?}
        \]

        Можно сделать так:

        $\cdot$ рассмотрим функцию 
        \begin{align*}
            &f(x) = x + x^2 + \dots + x^n \implies \\
            \implies &f'(x) = 1 + 2x + \dots + n x^{n - 1}
        \end{align*}
        / производная /

        Легко видеть, что $S = f'\left(\frac{1}{3}\right)$

        Но $f(x) = x + x^2 + \dots + x^n = 
        x(\underbrace{1 + \dots + x^{n - 1}}_\text{сумма геометрической прогрессии}) 
        = x \frac{1 - x^n}{1 - x}$
        
        Тогда $f'\left(\frac{1}{3}\right) 
        = (x \frac{1 - x^n}{1 - x})' 
        = \frac{(nx - n - 1) x^n + 1}{(x - 1)^2} 
        \implies f'\left(\frac{1}{3}\right) = \boxed{\frac{3^{-n + 1} (3^{n + 1} - 2n - 3)}{4} = S}$

        Если есть сомнения в корректности этого метода, то полученную формулу можно проверить по индукции
    \end{enumerate}
\end{document}
