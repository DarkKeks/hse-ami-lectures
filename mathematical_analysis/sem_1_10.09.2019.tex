\section{Семинар 1}

\subsection{Математическая индукция}
\[
	P_1, P_2, \dots, P_n, \dots
\]

Если 1) верно, 2) \( \forall n : P_n \rightarrow P_{n + 1} \implies P_n \) верно \( \forall n \) 

\( P_1 : \quad 3 | 1^1 - 1 \) \textit{верно}

\( P_n : \quad 3 | n^3 - n \)

\( P_{n + 1} : \quad 3 | (n + 1)^3 - (n + 1) \) ?
\begin{align*}
	(n + 1)^3 - (n + 1) &= n^3 + 3n^2 + 3n - n - 1 \\
	&= n^3 - n + 3(n^2 + n)
\end{align*}

\begin{problem}
	\( 1^2 + 2^2 + \dots + n^2 = \frac{n(n + 1)(2n + 1)}{6} \)
\end{problem}

\( P_1: 1^2 = \frac{1(1+1)(2+1)}{6} = 1\)

Шаг: \( P_n \) верно, тогда
\begin{align*}
	P_{n + 1} : 1^2 + 2^2 + \dots + n^2 + (n + 1)^2 
	&= \frac{n(n + 1)(2n + 1)}{6} + (n + 1)^2 \\
	&= \frac{n(n + 1)(2n + 1) + 6(n + 1)^2}{6} \\
	&= \frac{(n + 1)(2n^2 + n + 6n + 6)}{6} \\
	&= \frac{(n + 1)(2n^2 + 7n + 6)}{6} \\
	&= \frac{(n + 1)(2n + 3)(n + 2)}{6} \\
	&= \frac{(n + 1)((n + 1) + 1)(2(n + 1) + 1)}{6}
\end{align*}

\subsection{Неравенство бернулли}

\begin{problem}
	\( 
		\forall n \in \NN, 
		\forall x \in [-1; +\infty) 
		\implies (1 + x)^n \geq 1 + nx 
	\)
\end{problem}
\begin{alignat*}{3}
	&P_1 : (1 + x) \geq 1 + 1 x \\
	&P_{n} : (1 + x)^{n} \geq 1 + nx 
	\xrightarrow{(1 + x) \geq 0}
	&&(1 + x)^{n + 1} && \geq (1 + nx)(1 + x) \\
	& &&(1 + x)^{n + 1} && \geq 1 + (n + 1)x + nx^2 \\
	& &&(1 + x)^{n + 1} && \geq 1 + (n + 1)x	
\end{alignat*}

\begin{problem}
	\( \forall n \in \NN, \forall x \in [-2; +\infty) \implies (1 + x)^n \geq 1 + nx \)
\end{problem}

\( P_1 : 1 + x \geq 1 + 1 x \)

\( P_{n} : (1 + x)^{n} \geq 1 + nx \)

\begin{align*}
	\underbracket{(1 + x)^{n + 1}} + \underbracket{(1 + x)^n}
	= (1 + x)^n(\underbrace{1 + x + 1}_{x + 2 \geq 0}) 
	\geq (1 + nx) (1 + x + 1)
	= 
	\underbracket{1 + (n + 1)x} + \underbracket{1 + nx(1 + x)}
\end{align*}

Для \( x \in [-2; -1) \) выполняется \( (1 + x)^n \leq 1 \leq 1 + n\underbrace{x(1 + x)}_{> 0} \)

\subsection{Биноминальные коэффициенты}

\[
	\binom{n}{k} = C_n^k = \frac{n!}{(n - k)! k!}
\]

\[
	\binom{n}{k} + \binom{n}{k + 1} = \binom{n + 1}{k + 1}
\]

% Можно тут триугольник паскаля нарисовать, показав сумму

\subsubsection{Бином Ньютона}

\begin{problem} 
	Доказать для \( a, b \neq 0 \)
	\begin{align*}
		(a + b)^n 
		&= \binom{n}{0}a^n b^0 + \binom{n}{1} a^{n - 1} b^1 + \binom{n}{2} a^{n - 2} b^2 + \dots 
			+  \binom{n}{n - 1} a b^{n-1} + \binom{n}{n} a^0 b^n \\
		&= \sum^n_{k=0} \binom{n}{k} a^{n - k} b^k
	\end{align*}
\end{problem}

Пусть \( x = \frac{a}{b}, x \in R \)
\begin{align*}
	\frac{(a + b)^n}{b^n} = (1 + x)^n &= \sum^n_{k = 0} \binom{n}{k} x^k 
	= \sum_{k = 0}^n \binom{n}{k} \frac{a^k}{b^k} \implies \\
	\implies (a + b)^n &= \sum_{k=0}^n \binom{n}{k} a^k b^{n - k}
\end{align*}


\begin{problem}
	Доказать \((1 + x)^n = \sum^n_{k = 0} \binom{n}{k} x^k \)
\end{problem}
Докажем по индукции
\begin{align*}
	P_1 : (1 + x)^1 &= 1 + x = \binom{1}{0}1 x^0 + \binom{1}{1} x^1 \\
	P_{n+1} : (1 + x)^{n + 1} &=
	(1 + x)^n (1 + x) = (1 + x)^n + (1 + x)^n x \\
	&= x^0 + \binom{n}{1} x^1 + \binom{n}{2} x^2 + \dots + \binom{n}{n}x^n + \\
	&\qquad + \binom{n}{0} x^1 + \binom{n}{1} x^2 + \dots + \binom{n}{n - 1}x^n + x^{n + 1} \\
	&= \binom{n + 1}{0} x^0 + \left(\binom{n}{1} + \binom{n}{0} \right) x^1 + \left(\binom{n}{2} + \binom{n}{1} \right) x^2 + \dots + \left( \binom{n}{n} + \binom{n}{n - 1} \right) x^n + x^{n + 1} \\ 
	&= \binom{n + 1}{0} x^0 + \binom{n + 1}{1} x^1 + \binom{n + 1}{2} x^2 + \dots + \binom{n + 1}{n} x^n + \binom{n + 1}{n + 1} x^{n + 1}
\end{align*}

\begin{problem}
	Доказать \( 2^n  = \sum_{k = 0}^n \binom{n}{k} \)
\end{problem}
\[ (1 + x)^n = \sum^n_{k = 0} \binom{n}{k} x^k \]
\[ \sum^n_{k = 0} \binom{n}{k} 1 = 2^n \]
\[ x = 1 \implies 2^n = (1 + 1)^n = \sum^n_{k=0} \binom{n}{k} 1^k \]

\subsection{Сумма геометрической прогрессии}

\[ \sum_{k=1}^n a q^{k - 1} = S_n \]
\begin{align*}
	S_n q = \sum_{k = 1}^n a q^k &= S_n - a + a q^n \\
	&= \underbrace{a + aq + aq^2 + \dots + aq^{n-1}}_{S_n} + aq^n - a \implies \\
	\implies (q - 1) S_n &= a q^n - a \implies \\
	\implies S_n &= 
	\begin{cases}
		a \frac{q^n - 1}{q - 1}, & q \neq 1 \\
		na, & q = 1
	\end{cases}
\end{align*}

\subsection{Рациональные числа с двумя десятичными представлениями}

\( \pi = 3.1415... \)

\( \frac{1}{3} = 0.3333...\)

\( 1 = 1.00000...{}0... = 0.9999999... \)

\( 
	\frac{9}{10} + \frac{9}{10^2} + \frac{9}{10^3} + \dots  
	= \frac{9}{10} (1 + \frac{1}{10} + \frac{1}{10^2} + \dots)
	= \frac{9}{10}\frac{10}{9} = 1
\)


