\section{Лекция 4}

\begin{statement}
    $\{a_n\}$ ограничена сверху. 

    $a = \lim x_n$
    
    $\exists M : \forall n \implies x_n \leq M$
\end{statement}

\begin{proof}
    Пусть $a > M$

    $a = \lim_{n \to \infty} x_n$ -- $\forall \epsilon > \exists N : \forall n > N \implies |x_n - a| < \epsilon$

    Распишем модуль
    $a - \epsilon < x_n < a + \epsilon$

    Наше предположение: $a > M$

    Отсюда сразу следует, что $a - \epsilon < x_n \leq M$

    Возьмем $\epsilon$ такое, что $a - \epsilon > M$, например $\epsilon = \frac{a - M}{2}$

    Тогда, начиная с некоторого $N$ все точки после $x_N$ будут лежать в $\epsilon$-окресности точки $a$.

    $\forall n > N \implies x_n \geq M$
\end{proof}

\begin{statement}
    $a = \lim_{n \to \infty} x_n$, $b = \lim_{n \to \infty} x_n$
    Тогда, $a = b$.
\end{statement}

\begin{proof}
   Допустим $a < b$ 

   $\forall \epsilon > 0 \ \exists N_1 : \forall n > N_1 \implies |x_n - a| < \epsilon$
    
   $\forall \epsilon > 0 \ \exists N_2 : \forall n > N_2 \implies |x_n - b| < \epsilon$

    Выбираем $N = max(N_1, N_2)$

    Тогда, для $N$ будут справедливы оба неравенства.

    Например, пусть $\epsilon = \frac{b - a}{2}$, тогда с одной стороны 
    \begin{equation*}
        \forall n > N \implies x_n < a + \epsilon < \frac{a + b}{2}  
    .\end{equation*}
    а с другой стороны 
    \begin{equation*}
        \forall n > N \implies x_n > b - \epsilon > \frac{a + b}{2}
    .\end{equation*}
\end{proof}

\begin{statement}
    $\lim_{n \to \infty} x_n = a$

    $\lim_{n \to \infty} y_n = b$

    \begin{enumerate}
        \item $\lim_{n \to \infty} (x_n + y_n) = a + b$
        \item $\lim_{n \to \infty} x_n y_n = a \cdot b$
        \item $\lim_{n \to \infty} \frac{x_n}{y_n} = \frac{a}{b}$

    \end{enumerate}
\end{statement}

\begin{proof}~
    \begin{enumerate}
    \item 
        Если есть есть беконечно малые последовательности, то их сумма тоже бесконечно малая.
        
        $\alpha_n$ -- бесконечно малая
        
        $\beta_n$ -- бесконечно малая

        $\alpha_n + \beta_n$ -- бесконечно малая

        $\lim_{n \to \infty} \alpha_n = 0$ --- $\forall \epsilon > 0 \ \exists N_1 : \forall n > N_1 \implies |\alpha_n| < \frac{\epsilon}{2}$

        $\lim_{n \to \infty} \beta_n = 0$ --- $\forall \epsilon > 0 \ \exists N_2 : \forall n > N_2 \implies |\beta_n|< \frac{\epsilon}{2}$

        Пусть $N = max(N_1, N_2)$

        $\forall \epsilon > 0 \ \exists N : \forall n > N \implies |\alpha_n + \beta_n| \leq |\alpha_n| + |\beta_n| < \epsilon$
    
        $\lim_{n \to \infty} x_n = a$ --- $\forall \frac{\epsilon}{C} > 0 \ \exists N : \forall n > N \implies |x_n - a| < C \cdot \frac{\epsilon}{C} = \epsilon$

        $\{\alpha_n\}$ -- бесконечно малая

        $\{z_n\}$ -- ораниченная
        ($\exists M > 0 : \forall n \implies \{z_n\} = M$)

        $\{\gamma_n\} = \{\alpha_n \cdot z_n\}$ -- бесконечно малая

        $\forall \epsilon > 0 \ \exists N : \forall n > N \implies |\alpha_n| < \epsilon$

        $|\gamma_n| = |\alpha_n z_n| = |z_n||\alpha_n| \leq M \cdot \epsilon$

        $a = \lim_{n \to \infty} x_n \implies x_n = a + \alpha_n$

        $b = \lim_{n \to \infty} y_n \implies y_n = b + \beta_n$

        $x_n + y_n = a + b + (\alpha_n + \beta_n)$

        $\mu_n = \alpha_n + \beta_n$ -- бесконечно малая
    \item
        $x_n \cdot y_n = (a + \alpha_n)(b + \beta_n) = a \cdot b + (\alpha_n \cdot b + \beta_n \cdot a + \alpha_n \cdot \beta_n) = a \cdot b + \mu_n$

        $\mu_n = \alpha_n \cdot b + \beta_n \cdot a + \alpha_n \cdot \beta_n$

    \item
        $\frac{x_n}{y_n} = \frac{a + \alpha_n}{b + \beta_n} = \frac{a}{b} + \underbrace{\frac{a + \alpha_n}{b + \beta_n} - \frac{a}{b}}_{\mu_n}$

        $\mu_n = \frac{a + \alpha_n}{b + \beta_n} + \frac{a}{b} = \frac{a \cdot b + \alpha_n \cdot b - a \cdot b - a \beta_n}{b(b + \beta_n)} = \alpha_n \cdot b + \beta_n \cdot a + \frac{1}{b(b + \beta_n}$

        $|b + \beta_n| = |b - (-\beta_n)| \geq ||b| - |\beta_n||$

        $\beta_n$ -- бесконечно малая

        $\forall \epsilon > 0 \ \exists N : \forall n > N \implies |b_n| < \epsilon$

        $\epsilon = \frac{|b|}{2}$

        $|b + \beta_n| > \frac{|b|}{2}$

        $\frac{1}{|b + \beta_n|} < \frac{2}{|b|}$

        $\frac{x_n}{y_n} = \frac{a}{b} + \mu_n$, $\mu_n$ -- бесконечно малая

        $\lim_{n \to \infty} \frac{x_n}{y_n} = \frac{a}{b}$
    \end{enumerate}
\end{proof}


\begin{theorem}
    Монотонная ограниченная последовательность сходится.
\end{theorem}

\begin{enumerate}
\item $\forall n \implies x_n < x_{n + 1}$
\item $\exists M : \forall n \implies x_n \leq M$
\end{enumerate}

\begin{proof}
    \begin{enumerate}
    \item
        $\{x_n\}$ ограничено сверху

        $\exists M = \sup \{x_n\}$

        $M \overset{?}{=} \lim_{n \to \infty} x_n$

        $\forall n \implies x_n \leq M$

        $\forall e > 0 \ \exists \overline{x}_n \in \{x_n\} : \overline{x}_n > M - \epsilon$
        
        $\overline{x}_n = x_{n_0} > M - \epsilon$
    \end{enumerate}
\end{proof}

