\documentclass[a4paper]{article}

\usepackage[T2A]{fontenc}
\usepackage[utf8]{inputenc}
\usepackage[english, russian]{babel}

\usepackage[top=0.8in,bottom=0.8in,left=0.8in,right=0.8in,headheight=110pt]{geometry}
\usepackage{mathtools, amsthm, amssymb}
\usepackage{indentfirst}
\usepackage{enumitem}
\usepackage[unicode=true, colorlinks=true, linkcolor=blue]{hyperref}

\usepackage{array}

\newcommand{\NN}{\mathbb{N}}
\newcommand{\ZZ}{\mathbb{Z}}
\newcommand{\QQ}{\mathbb{Q}}

\theoremstyle{definition}
\newtheorem{definition}{Определение}
\newtheorem*{consequence}{Следствие}
\newtheorem*{example}{Пример}

\theoremstyle{remark}
\newtheorem*{exercise}{Упражнение}
\newtheorem*{statement}{Утверждение}
\newtheorem*{remark}{Примечание}
\newtheorem*{answer}{Ответ}
\newtheorem*{hint}{Подсказка}

\theoremstyle{plain}
\newtheorem*{lemma}{Лемма}
\newtheorem{theorem}{Теорема}
\newtheorem{problem}{Задача}

% \everymath{\displaystyle}

\title{Лекции по математическому анализу}
\author{Slava Boben}
\date{September 5, 2019}

\begin{document}
	\pagestyle{empty}
	\maketitle
	\tableofcontents
	\newpage
	
	\pagestyle{plain}
	
	\section{Лекция 1}

\subsection{Общая информация}
\subsubsection{Контакты}
Делицын Андрей Леонидович
\begin{itemize}[noitemsep]
	\item delitsyn@mail.ru
	\item adelistyn@hse.ru
\end{itemize}

\subsubsection{Оценка}
\[ 
	O_\text{пром} = \text{Округление}\left( 
		\frac{1}{10} \text{ДЗ}_1 + 
		\frac{1}{10} \text{КР}_1 + 
		\frac{3}{10} \text{КК}_1 + 
		\frac{5}{10} \text{Экз}_1
	\right)
\]
\[
	O_\text{итог} = \text{Округление}\left(
		\frac{1}{20} \text{ДЗ}_2 + 
		\frac{1}{20} \text{КР}_2 + 
		\frac{3}{20} \text{КК}_2 +
		\frac{5}{20} \text{Экз}_2 + 
		\frac{10}{20} O_\text{пром} 
	\right)
\]
\[
	\text{Округление}\left(x\right) = \left[ x \right] 
\]

\subsubsection{Коллоквиум}
\begin{description}
	\item [1 балл] Простая задача (например, продифференцировать функцию)
\end{description}

Основная часть

На подготовку к основной части дается 40 минут
\begin{description}
	\item [1 балл] Формулировка теоремы без доказательства или определение некоторого понятия
	\item [2 балла] Решить задачу
	\item [4 балла] Доказательство теоремы
	\item [2 балла] Дополнительные вопросы
\end{description}

\subsubsection{Экзамен}
То же что и коллоквиум, за исключением первой части (один балл с нее идет в второе задание)
\subsubsection{Литература}
\begin{enumerate}
	\item А. М. Тер-Крикоров, М. И. Шабунин. Курс математического анализа: учеб. пособие для вузов, М.:	МФТИ, 2000.
	
	\item Фихтенгольц Г.М. Курс дифференциального и интегрального исчисления, СПб.: Лань, 2001.
	
	\item Демидович Б.П. Сборник задач и упражнений по математическому анализу: учеб. пособие для вузов, М.: АСТ: Астрель, 2004.
	
	\item Л. Д. Кудрявцев [и др.]; Под ред. Л. Д. Кудрявцева. Сборник задач по математическому анализу: в 3 т.
	
	\item Зорич, В. А. Математический анализ: учебник, М.: МЦНМО, 2015.
\end{enumerate}

\subsection{Теория пределов и непрерывных функций одной переменной}

\subsubsection{Понятие действительного числа}
\begin{enumerate}[label=(\alph*), series=numbers]
	\item \emph{Натуральные числа}
	\[ \mathbb{N} \text{ -- } 1, 2, \dots, n, \dots \]
	\[ +, \times \]
	
	\item \emph{Целые числа}
	\[ \mathbb{Z} \text{ -- } {-1}, 0, 1, 2, \dots, n, \dots \]
	\[ +, -, \times \]
\end{enumerate}	

Группа - множество, в котором определена операция, такая что,
\begin{align*}
	&\forall a \in \ZZ, \forall b \in \ZZ, \quad \exists c \in \ZZ \\
	&c = a + b
\end{align*}

Свойства групп:
\begin{enumerate}[series=group_properties]
	\item \( (a + b) + d = a + (b + d) \)
	\item \( \exists 0 : a + 0 = a \)
	\item 
	\( \forall a \ \exists (-a) \) 
	
	\( a + (-a) = 0 \)
	\item \( a + b = b + a \) --- Коммутативная или абелевая
\end{enumerate} 
	
\begin{enumerate}[numbers]
	\item \emph{Рациональные числа}
	\begin{align*}
		&\QQ = \frac{m}{n} \\
		&m \in \NN, n \in \ZZ
	\end{align*}
	
	Операция сложения для рациональных цисел:
	\[
	\frac{m}{n} + \frac{k}{l} = \frac{ml + nk}{nl}
	\]
\end{enumerate}
	
Пусть есть два числа, \( \forall a \in \mathbb{Q}, \forall b \in \mathbb{Q} \), для этих чисел определена операция умножения
\begin{enumerate}[group_properties]
	\item \( (a \times b) \times c = a \times (b \times c) \)
	\item \( \exists 1 : a \times 1 = a \)
	\item 
	\( \forall a \ \exists a^{-1} \) 
	
	\( a \times a^{-1} = 1 \)
	\item \( a \times b = b \times a \)
	\item \( a \times (b + c) = a \times b + a \times c \)
\end{enumerate}
	
Таким образом рациональные числа -- это группа по сложение и за исключением нуля по умножению. 
Если выполняются все 9 свойств --- соответствующее множество называется полем.
Возникает вопрос, достаточно ли рациональных цифр?

Вспомним теорему Пифагора --- \( a^2 + b^2 = c^2 \). Допустим, \( a = 1, b = 1 \), тогда \( c^2 = 2 \). Для нас важно то, что это число \( c \) не может быть представлено никакой рациональной дробью, давайте проверим это.

Предположим что есть \( c = \frac{m}{n} \), где \( m \in \NN, n \in \ZZ \). Можно считать что дробь \( \frac{m}{n} \) несократима.
\[
	c^2= 2 \implies \frac{n^2}{m^2} = 2 \implies m^2 = 2n^2 \\
\]
\( m \) -- четное, \( m = 2p \)
\[
	4p^2 = 2n^2 \implies n^2 = 2p^2
\]
\( n \) -- тоже четное, \( n = 2k \) \\

Тогда поделим \( n \) на \( m \): \( \frac{m}{n} = \frac{2p}{2k} \), дробь сократима, пришли к противоречию

Получается, множества рациональных чисел не хватает чтобы взять корень из двойки.

И вот теперь, мы сделаем следующее построение, позволяющее ввести понятие вещественных чисел, и вообще говоря, измерить любой отрезок на прямой

Будет ли это множество полем? Будет, и более того, это множество позволит нам решить проблему измерения любого отрезка. А есть ли какие либо поля, которые содержат в себе поле вещественных чисел? Есть, таким полем является поле комплексных чисел. Все ли свойства вещественых чисел имеют место для комплексных чисел? Нет, вещественные числа всегда можно сравнивать. 

\newpage

\( \forall a, b \)
\begin{enumerate}
	\item \(
		\begin{cases*}
			a < b \\
			b < a \\
			a = b
		\end{cases*}
	\)
	\item \( a < b, b < c \implies a < c \)
	\item \(a < b \implies \forall c : a + c < b + c \)
	\item \( \forall c > 0 : a * c < b * c \)
\end{enumerate}
	\section{Семинар 1}

\subsection{Математическая индукция}
\[
	P_1, P_2, \dots, P_n, \dots
\]

Если 1) верно, 2) \( \forall n : P_n \rightarrow P_{n + 1} \implies P_n \) верно \( \forall n \) 

\( P_1 : \quad 3 | 1^1 - 1 \) \textit{верно}

\( P_n : \quad 3 | n^3 - n \)

\( P_{n + 1} : \quad 3 | (n + 1)^3 - (n + 1) \) ?
\begin{align*}
	(n + 1)^3 - (n + 1) &= n^3 + 3n^2 + 3n - n - 1 \\
	&= n^3 - n + 3(n^2 + n)
\end{align*}

\begin{problem}
	\( 1^2 + 2^2 + \dots + n^2 = \frac{n(n + 1)(2n + 1)}{6} \)
\end{problem}

\( P_1: 1^2 = \frac{1(1+1)(2+1)}{6} = 1\)

Шаг: \( P_n \) верно, тогда
\begin{align*}
	P_{n + 1} : 1^2 + 2^2 + \dots + n^2 + (n + 1)^2 
	&= \frac{n(n + 1)(2n + 1)}{6} + (n + 1)^2 \\
	&= \frac{n(n + 1)(2n + 1) + 6(n + 1)^2}{6} \\
	&= \frac{(n + 1)(2n^2 + n + 6n + 6)}{6} \\
	&= \frac{(n + 1)(2n^2 + 7n + 6)}{6} \\
	&= \frac{(n + 1)(2n + 3)(n + 2)}{6} \\
	&= \frac{(n + 1)((n + 1) + 1)(2(n + 1) + 1)}{6}
\end{align*}

\subsection{Неравенство бернулли}

\begin{problem}
	\( 
		\forall n \in \NN, 
		\forall x \in [-1; +\infty) 
		\implies (1 + x)^n \geq 1 + nx 
	\)
\end{problem}
\begin{alignat*}{3}
	&P_1 : (1 + x) \geq 1 + 1 x \\
	&P_{n} : (1 + x)^{n} \geq 1 + nx 
	\xrightarrow{(1 + x) \geq 0}
	&&(1 + x)^{n + 1} && \geq (1 + nx)(1 + x) \\
	& &&(1 + x)^{n + 1} && \geq 1 + (n + 1)x + nx^2 \\
	& &&(1 + x)^{n + 1} && \geq 1 + (n + 1)x	
\end{alignat*}

\begin{problem}
	\( \forall n \in \NN, \forall x \in [-2; +\infty) \implies (1 + x)^n \geq 1 + nx \)
\end{problem}

\( P_1 : 1 + x \geq 1 + 1 x \)

\( P_{n} : (1 + x)^{n} \geq 1 + nx \)

\begin{align*}
	\underbracket{(1 + x)^{n + 1}} + \underbracket{(1 + x)^n}
	= (1 + x)^n(\underbrace{1 + x + 1}_{x + 2 \geq 0}) 
	\geq (1 + nx) (1 + x + 1)
	= 
	\underbracket{1 + (n + 1)x} + \underbracket{1 + nx(1 + x)}
\end{align*}

Для \( x \in [-2; -1) \) выполняется \( (1 + x)^n \leq 1 \leq 1 + n\underbrace{x(1 + x)}_{> 0} \)

\subsection{Биноминальные коэффициенты}

\[
	\binom{n}{k} = C_n^k = \frac{n!}{(n - k)! k!}
\]

\[
	\binom{n}{k} + \binom{n}{k + 1} = \binom{n + 1}{k + 1}
\]

% Можно тут триугольник паскаля нарисовать, показав сумму

\subsubsection{Бином Ньютона}

\begin{problem} 
	Доказать для \( a, b \neq 0 \)
	\begin{align*}
		(a + b)^n 
		&= \binom{n}{0}a^n b^0 + \binom{n}{1} a^{n - 1} b^1 + \binom{n}{2} a^{n - 2} b^2 + \dots 
			+  \binom{n}{n - 1} a b^{n-1} + \binom{n}{n} a^0 b^n \\
		&= \sum^n_{k=0} \binom{n}{k} a^{n - k} b^k
	\end{align*}
\end{problem}

Пусть \( x = \frac{a}{b}, x \in R \)
\begin{align*}
	\frac{(a + b)^n}{b^n} = (1 + x)^n &= \sum^n_{k = 0} \binom{n}{k} x^k 
	= \sum_{k = 0}^n \binom{n}{k} \frac{a^k}{b^k} \implies \\
	\implies (a + b)^n &= \sum_{k=0}^n \binom{n}{k} a^k b^{n - k}
\end{align*}


\begin{problem}
	Доказать \((1 + x)^n = \sum^n_{k = 0} \binom{n}{k} x^k \)
\end{problem}
Докажем по индукции
\begin{align*}
	P_1 : (1 + x)^1 &= 1 + x = \binom{1}{0}1 x^0 + \binom{1}{1} x^1 \\
	P_{n+1} : (1 + x)^{n + 1} &=
	(1 + x)^n (1 + x) = (1 + x)^n + (1 + x)^n x \\
	&= x^0 + \binom{n}{1} x^1 + \binom{n}{2} x^2 + \dots + \binom{n}{n}x^n + \\
	&\qquad + \binom{n}{0} x^1 + \binom{n}{1} x^2 + \dots + \binom{n}{n - 1}x^n + x^{n + 1} \\
	&= \binom{n + 1}{0} x^0 + \left(\binom{n}{1} + \binom{n}{0} \right) x^1 + \left(\binom{n}{2} + \binom{n}{1} \right) x^2 + \dots + \left( \binom{n}{n} + \binom{n}{n - 1} \right) x^n + x^{n + 1} \\ 
	&= \binom{n + 1}{0} x^0 + \binom{n + 1}{1} x^1 + \binom{n + 1}{2} x^2 + \dots + \binom{n + 1}{n} x^n + \binom{n + 1}{n + 1} x^{n + 1}
\end{align*}

\begin{problem}
	Доказать \( 2^n  = \sum_{k = 0}^n \binom{n}{k} \)
\end{problem}
\[ (1 + x)^n = \sum^n_{k = 0} \binom{n}{k} x^k \]
\[ \sum^n_{k = 0} \binom{n}{k} 1 = 2^n \]
\[ x = 1 \implies 2^n = (1 + 1)^n = \sum^n_{k=0} \binom{n}{k} 1^k \]

\subsection{Сумма геометрической прогрессии}

\[ \sum_{k=1}^n a q^{k - 1} = S_n \]
\begin{align*}
	S_n q = \sum_{k = 1}^n a q^k &= S_n - a + a q^n \\
	&= \underbrace{a + aq + aq^2 + \dots + aq^{n-1}}_{S_n} + aq^n - a \implies \\
	\implies (q - 1) S_n &= a q^n - a \implies \\
	\implies S_n &= 
	\begin{cases}
		a \frac{q^n - 1}{q - 1}, & q \neq 1 \\
		na, & q = 1
	\end{cases}
\end{align*}

\subsection{Рациональные числа с двумя десятичными представлениями}

\( \pi = 3.1415... \)

\( \frac{1}{3} = 0.3333...\)

\( 1 = 1.00000...{}0... = 0.9999999... \)

\( 
	\frac{9}{10} + \frac{9}{10^2} + \frac{9}{10^3} + \dots  
	= \frac{9}{10} (1 + \frac{1}{10} + \frac{1}{10^2} + \dots)
	= \frac{9}{10}\frac{10}{9} = 1
\)



    \section{Понятие предела числовой последовательности}

$x_1 = 1, x_2 = \frac{1}{2}, \dots, x_n = \frac{1}{n}, \dots$

$\lim_{x \to \infty} x_n = 0$

\begin{definition}
    $x = \lim_{x\to\infty} x_n \quad \forall \epsilon > 0 \quad \exists N, \forall n > N \implies |x_n - a| < \epsilon$

    $(a - \epsilon, a + \epsilon)$ -- $\epsilon$ окрестность точки $a$.
\end{definition}

\begin{definition}
    $x_n$ -- называется бесконечно малым, если
    $lim_{x\to\infty} x_n = 0$
\end{definition}

\begin{theorem}
    $a = \lim_{x\to\infty} x_n \implies x_n = a + \alpha_n$, $\alpha_n$ -- бесконечно малая последовательность

    $\forall \epsilon > 0 \quad \exists N \quad \forall n > N \implies |x_n - a| < \epsilon_n$

    $\alpha_n = x_n - a$

    $x_n = a + \alpha_n$

    $\lim_{n\to\infty} \alpha_n = 0$
\end{theorem}

\begin{definition}
    $\{x_n\}$ (последовательность) называется ограниченной, если $\exists M > 0 \quad \forall n \implies |x_n| \leq M$
\end{definition}

\begin{definition}
    $\{x_n\}$ (последовательность) называется неограниченной, если $\forall M > 0 \quad \exists n_0 \implies |x_n| > M$
\end{definition}

\begin{definition}
    $\{x_n\}$ (последовательность) называется бесконечно большой, если $\forall M > 0 \quad \exists N \quad \forall n > N \implies |x_n| > M$
\end{definition}


% утверждение
\begin{statement}
    Сходящиеся последовательности ограничены
\end{statement}

\begin{statement}
    $a = \lim_{n\to\infty} x_n \implies \exists M > 0 \quad \forall n \in \NN \implies |x_n| \leq M$
\end{statement}

	\section{Семинар 2}
\subsection{Числовые последовательности}

$n \mapsto a_n, n \in N$

$f: \NN \to \RR$

$n \mapsto f(n)$

\subsubsection{Общие понятия}
\begin{enumerate}
\item
    Ограничена/не ограничена -- $\exists M : \forall n \in \NN \implies a_n \leq M$

\item
    $\inf$ и $\sup$
    
    \begin{itemize}
    \item 
        $s - \sup$ для $\{a_n\}_1^\infty$, если 

        \begin{enumerate}
        \item 
            $\forall n : a_n \leq s$
        \item
            $\forall s' < s \exists n \in \NN \implies s' < a_n$
        \end{enumerate}
    \item
        $i - \inf$ для $\{a_n\}_1^\infty$, если
        \begin{enumerate}
        \item $\forall n : a_n \geq i$
        \item $\forall i' > i \exists n \in \NN \implies a_n < i'$
        \end{enumerate}
    \end{itemize}

\item
    Монотонность

\item
    Существование предела: число $a$ называется пределом последовательности $\{a_n\}_1^\infty$, если 

    $\forall \epsilon > 0 \quad \exists N (\epsilon) \in \NN : \forall n \geq N \implies |a_n - A| < \epsilon$

\item 
    $\{a_n\}$ бесконечно большая, если $\forall E > 0 \quad \exists N \in \NN : \forall n \geq N \implies a_n > E$

    $\lim_{n\to\infty} a_n = +\infty$

\item 
    $\{a_n\}$ бесконечно малая, если $\forall \epsilon > 0 \quad \exists N : \forall n \geq N : |a_n - A| < \epsilon$
    
    $\lim_{n\to\infty} a_n = 0$

\item
    Монотонно возрастающая ограниченная сверху ? (Сходится к $\sup$)
\end{enumerate}

\subsection{Примеры}

\begin{enumerate}
    \item
        $\{(-1)^n\}$

        $-1 \leq a_n \leq 1 (\forall n)$

        $\inf\{a_n\} = -1$

        $\sup\{a_n\} = 1$

        Не монотонна

        Нет предела

    \item 
        $a_n = \sqrt{n + 2} - \sqrt{n + 1} > 0$

        Монотонность:

        $a_{n + 1} - a_n < 0$

        $\frac{a_{n + 1}}{a_n} < 1$

        $\frac{a_{n + 1}}{a_n} = \frac{\sqrt{n + 3} - \sqrt{n + 2}}{\sqrt{n + 2} - \sqrt{n + 1}}
        = \frac{n + 3 - (n + 2)}{n + 2 - (n + 1)} \cdot \frac{\sqrt{n + 2} + \sqrt{n + 1}}{\sqrt{n + 3} + \sqrt{n + 2}}$
    
        Предел:

        $\lim_{n \to \infty} (\sqrt{n + 2} - \sqrt{n + 1})$

        $\lim_{n \to \infty} \left(\frac{n + 2 - (n + 1)}{\sqrt{n + 2} + \sqrt{n + 1}}\right) = 0$

        Является бесконечно малой (предел равен 0)

    \item 
        $\{\frac{n^2}{2^n}\}_n^\infty = \{\frac{1}{2}, \frac{4}{4}, \frac{9}{8}, \frac{16}{16}, \frac{25}{32}, \dots\}$

        $\frac{a_{n + 1}}{a_n} = \frac{(n + 1)^2}{2^n \cdot 2} \cdot \frac{2^n}{n^2} = \frac{1}{2}\left( 1 + \frac{1}{n} \right)^2 < 1 
        \leftrightarrow (1 + \frac{1}{n})^2 < 2
        \leftrightarrow 1 + \frac{1}{n} < \sqrt{2}$

        Монотонно убывает

        Предел:

        $\lim_{n \to \infty} \frac{n^k}{\lambda^n} = 0, \forall k \in \NN, \lambda > 1$

        (Экспонента растет принципиально быстрее чем любой многочлен)

        \begin{proof}~\\
            $k = 1$

            $\binom{n}{2} = \frac{n!}{2!(n - 2)!} = \frac{n(n - 1)}{2} \geq \frac{n^2}{4}$

            $n - 1 \geq \frac{n}{2} \quad (\forall n \geq 2)$
            
            $\lambda^n = (1 + \underbrace{\lambda - 1}_{> 0})^n
            = \sum_{i = 0}^{n} \binom{n}{i} (\lambda - 1)^i
            \overset{i = 2}{\geq} \binom{n}{2} (\lambda - 1)^2
            \overset{n \geq 2}{\geq} \underbracket{\frac{(\lambda - 1)^2}{4}}_\text{константа} \cdot n^2$

            $\lambda^n \geq \frac{(\lambda - 1)^2}{4} \cdot n^2$

            $\frac{n}{\lambda^n} \leq \frac{n}{\frac{(\lambda - 1)^2}{4}} \cdot \frac{1}{n^2} = \frac{4}{(\lambda - 1)^2} \cdot \frac{1}{n} \to 0$
        \end{proof}

    \item
        $\{\frac{2^n}{n!}\}_{n = 1}^\infty \to 0, n \to +\infty$

        $n \geq 4$

        $\frac{2^n}{n!} = \frac{2 \cdot 2 \cdot 2 \cdot 2 \cdot 2 \cdot \dots \cdot 2}{1 \cdot 2 \cdot 3 \cdot 4 \cdot 5 \cdot \dots \cdot n} 
        \leq \frac{8}{6} \cdot \left( \frac{1}{2} \right)^{n - 3}$

    \item
        $a_n < b_n \quad \forall n \implies \lim_{n \to \infty} a_n \leq \lim_{n \to \infty} b_n $

    \item
        \begin{theorem}
            $\forall n : a_n \leq b_n \leq c_n$

            $\lim_{n \to \infty} a_n = \lim_{n \to \infty} c_n \implies \lim_{n \to \infty} b_n = \lim_{n \to \infty} a_n$
        \end{theorem}

    \item
        $\{ a_n \}_{n = 2}^\infty = \frac{1}{1 \cdot 2} + \frac{1}{2 \cdot 3} + \dots + \frac{1}{(n - 1) \cdot n}
        = \left(\frac{1}{1} - \frac{1}{2}\right) + \left(\frac{1}{2} - \frac{1}{3}\right) + \dots + \left(\frac{1}{n - 1} + \frac{1}{n}\right) 
        = \frac{1}{1} - \frac{1}{n}$

        $\frac{1}{(k - 1)k} = \frac{1}{k - 1} - \frac{1}{k}$
         

        $\lim_{n \to \infty} \{ a_n \} \to 1$

    \item 
        $\lim_{n \to \infty} \left(\sqrt[3]{3} \cdot \sqrt[3^2]{3} \cdot \sqrt[3^3]{3} \cdot \dots \cdot \sqrt[3^n]{3} \right)
        = 3^{\frac{1}{3}} \cdot 3^{\frac{1}{3^2}} \cdot \dots \cdot 3^{\frac{1}{3^n}} 
        = 3^{\frac{1}{3} + \frac{1}{3^2} + \dots + \frac{1}{3^n}} 
        = 3^{\frac{1 \cdot (1 - \frac{1}{3^n})}{3 \cdot (1 - \frac{1}{3})}} = 3^{\frac{1}{3} \frac{1 - \lim_{n \to \infty} \frac{1}{3^n}}{1 - \frac{1}{3}}}
        = 3^{\frac{1 \cdot 3}{3 \cdot 2}} 
        = 3^{\frac{1}{2}}$
\end{enumerate}

\end{document}
