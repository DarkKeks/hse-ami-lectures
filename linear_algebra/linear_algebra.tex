\documentclass[a4paper]{article}

\usepackage[T2A]{fontenc}
\usepackage[utf8]{inputenc}
\usepackage[english, russian]{babel}

\usepackage[top=0.8in,bottom=0.8in,left=0.8in,right=0.8in,headheight=110pt]{geometry}
\usepackage{mathtools, amsthm, amssymb}
\usepackage{indentfirst}
\usepackage{enumitem}
\usepackage[unicode=true, colorlinks=true, linkcolor=blue]{hyperref}

\newcommand{\NN}{\mathbb{N}}
\newcommand{\ZZ}{\mathbb{Z}}
\newcommand{\QQ}{\mathbb{Q}}
\newcommand{\RR}{\mathbb{R}}

\theoremstyle{definition}
\newtheorem{definition}{Определение}
\newtheorem*{consequence}{Следствие}
\newtheorem*{example}{Пример}

\theoremstyle{remark}
\newtheorem*{exercise}{Упражнение}
\newtheorem*{remark}{Примечание}
\newtheorem*{answer}{Ответ}
\newtheorem*{hint}{Подсказка}

\theoremstyle{plain}
\newtheorem*{lemma}{Лемма}
\newtheorem{theorem}{Теорема}
\newtheorem{problem}{Задача}

\title{Линейная алгебра и геометрия}
\author{Slava Boben}
\date{September 9, 2019}

\begin{document}
	\pagestyle{empty}
	\maketitle
	\tableofcontents
	\newpage
	
	\pagestyle{plain}
	
	\section{Лекция 1}

\subsection{Общая информация}
\subsubsection{Контакты}
Авдеев Роман Сергеевич
\begin{itemize}[noitemsep]
	\item suselr@yandex.ru
	\item ravdeev@hse.ru
\end{itemize}

\subsubsection{О дисциплине}
1 -- 4 модули

Письменный экзамен: 2, 4 модули
\subsubsection{Оценка}
\begin{enumerate}[nosep]
	\item Экзамен
	\item Коллоквиум
	\item Контрольная работа
	\item Больше ДЗ
	\item Работа на семинарах
	\item Бонус -- Задачи из листков
\end{enumerate}

\begin{equation*}
	O_\text{Итог} = \min(10, \text{Округление}(0.4 * O_\text{Экз} + 0.22 * O_\text{Колл} + 0.16 * O_\text{КР} + 0.16 * O_\text{ДЗ} + 0.08 * O_\text{Сем} + 0.08 * O_\text{Л}), 10) \\
\end{equation*}
\begin{equation*}
	\text{Округление}(x) = [x]
\end{equation*}

\subsubsection{Содержание курса}

\begin{enumerate}
	\item Начало алгебры --- 9 -- 10 занятий
	
	\begin{itemize}[nosep]
		\item Матрицы
		\item Системы линейных уравнений
		\item Определители
		\item Комплексные числа
	\end{itemize}

	\item Собственно линейная алгебра
	\begin{itemize}
		\item Вектороное пространство
	\end{itemize}
\end{enumerate}

\subsection{Матрицы}
\begin{definition}
	Матрица размера \( n \times m \) --- это прямоугольная таблица высоты \( m \) и ширины \( n \)
\end{definition}
\[ 
	A = 
	\begin{pmatrix}
		a_{11}  & a_{12} & \dots & a_{1n} \\
		a_{21} & a_{22} & \dots & a_{2n} \\
		\vdots & \vdots & \ddots & \vdots \\
		a_{m1} & a_{m2} & \dots & a_{mn}
	\end{pmatrix}
\]

\(a_ij\) -- элемент на пересечении i-й строки и j-го столбца

Краткая запись -- \( A = (a_{ij}) \)

Множество всех матриц размера \(m \times n \) с коэффициентами из \( \RR \) (множество всех действительных чисел) --- \( \text{Mat}_{n*m}(\RR) \) или \( \text{Mat}_{n*m} \)

\begin{definition}
	Две матрицы \( A \in \text{Mat}_{n \times m} \) и \( B \in \text{Mat}_{p \times q} \) называются \textit{равными}, если \( m = p, n = q \), и соответствующие элементы равны
\end{definition}

\begin{example}
	\(
		\begin{pmatrix}
			\circ & \circ & \circ \\ \circ & \circ & \circ
		\end{pmatrix}
		\neq
		\begin{pmatrix}
			\circ & \circ \\ \circ & \circ \\ \circ & \circ
		\end{pmatrix}
	\)
\end{example}
% Пример с матрицами 2x3 и 3x2

\subsubsection{Операции над матрицами}

\( A, B \in \text{Mat}_{m*n} \)

\begin{itemize}
	\item \emph{Сумма} \( A + B := (a_{ij} + b_{ij}) \)
	\item \emph{Произведение на скаляр} \( \alpha \in \RR \implies \lambda A := (\lambda a_{ij})\)
\end{itemize}

Свойства суммы и произведения на скаляр

\( \forall A, B, C  \in \text{Mat}_{m*n} \forall \lambda, \mu \in \RR \)
\begin{enumerate}[label=(\arabic*), nosep]
	\item \( A + B = B + A \) (коммутативность)
	\item \( (A + B) + C = A + (B + C)\) (ассоциативность)
	\item \( A + 0 = 0 + A = A\), где \( 0 = \) % матрица нулей
	\item \( A + (-A) = 0 \)
	
	\( -A \) -- Противоположная матрица
	\item \( (\lambda + \mu) A = \lambda A + \mu A \)
	\item \( \lambda (A + B) = \lambda A + \lambda B \)
	\item \( \lambda (\mu A) = \lambda \mu A \)
	\item \( 1 A = A \)
\end{enumerate}

\begin{exercise}
	Доказать эти свойства
\end{exercise}

\begin{remark}
	Из свойств (1) -- (8) следует, что \( \text{Mat}_{n*m}(\RR) \) является векторным пространством над \( \RR \)
\end{remark}

\subsubsection{\(\RR^n\)}
\( \RR^n := \{ (x_1, \dots, x_n) \mid x_i \in \RR \ \forall i = 1, \dots, n \} \)

\( \RR^1 = \RR \) \quad -- числовая прямая

\( \RR^2 \) \quad -- плоскость

\( \RR^3 \) \quad -- трехмерное пространство \\

Договоримся отождествлять \( \RR^n \) со столбцами высоты \( n \)

\( (x_1, \dots, x_n) \leftrightarrow \left( \begin{smallmatrix}
	x_1 \\ \vdots \\ x_n
\end{smallmatrix} \right) \text{ --- ``вектор столбец''} \)

\( \RR^n \leftrightarrow \text{Mat}_{n * m}(\RR) \)

\( \left[ x = \begin{pmatrix}
	x_1 \\ \vdots \\ x_n
\end{pmatrix} \in \RR^n, y = \begin{pmatrix}
	y_1 \\ \vdots \\ y_n
\end{pmatrix} \in \RR^n \right] \implies \left[ x = y \iff x_i = y_i \forall i \right] \)

\( x + y := \begin{pmatrix}
	x_1 + y_1 \\ \vdots \\ x_n + y_n
\end{pmatrix} \)

\( \lambda \in \RR \implies \lambda x_i := (\lambda x_1, \dots) \)

\subsubsection{Транспонирование}

\( A \in \text{Mat}_{m*n} = \begin{pmatrix}
	a_{11} & a_{12} & \dots & a_{1n} \\
	a_{21} & a_{22} & \dots & a_{2n} \\
	\vdots & \vdots & \ddots & \vdots \\
	a_{m1} & a_{m2} & \dots & a_{mn}
\end{pmatrix} \leadsto A^T \in \text{Mat}_{n*m} := \begin{pmatrix}
a_{11} & a_{21} & \dots & a_{m1} \\
a_{12} & a_{22} & \dots & a_{m2} \\
\vdots & \vdots & \ddots & \vdots \\
a_{1n} & a_{2n} & \dots & a_{mn}
\end{pmatrix} \)

\( A^T\) ---  Транспонированная матрица

Свойства:
\begin{enumerate}[label=(\arabic*), nosep]
	\item \( (A^T)^T = A \)
	\item \( (A + B)^T = A^T + B^T \)
	\item \( (\lambda A)^T = \lambda A^T \)
\end{enumerate}

\begin{example}
	\( \begin{pmatrix}
		x_1 & \dots & x_n
	\end{pmatrix}^T = \begin{pmatrix}
		x_1 \\ \vdots \\ x_n
	\end{pmatrix} \)
\end{example}
\begin{example}
	\( \begin{pmatrix}
	x_1 \\ \vdots \\ x_n
	\end{pmatrix}^T = \begin{pmatrix}
	x_1 & \dots & x_n
	\end{pmatrix} \)
\end{example}
\begin{example}
	\( \begin{pmatrix}
		1 & 2 \\ 3 & 4 \\ 5 & 6
	\end{pmatrix}^T = \begin{pmatrix}
		1 & 3 & 5 \\
		2 & 4 & 6
	\end{pmatrix} \)
\end{example}

\subsubsection{Умножение матриц}

\( A = (a_{ij})\)

%добавить = колонка, = столбец

\( A_{(i)} \) -- \(i\)-я строка матрицы \( A \)

\( A^{(j)}\) -- \(j\)-й  столбец матрицы \( A \)

\begin{enumerate}[label=(\arabic*)]
	\item 
		Частный случай: Произведение строки на столбец одинаковой длинны 
	
		\( \underbrace{(x_1, \dots, x_n)}_{1 \times n} \underbrace{\begin{pmatrix}
			x_1 \\ \vdots \\ x_n
		\end{pmatrix}}_{n \times 1} = x_1 * y_1 + \dots + x_n * y_n \)
		
		%Подписи 1*n и n * 1
	\item
		\( A \) - матрица размера \( m * n \)
		
		\( B \) - матрица размера \( n * p \)
		
		Кол-во строк матрицы \( A \) равно кол-ву столбцов матрицы \( B \) --- условие согласованности матриц
		
		\( AB := C \in \text{Mat}_{m*p}\), где \( C_{ij} = A_{(i)} B^{(j)}\)
\end{enumerate}

\begin{example}
	\( \begin{pmatrix}
		y_1 \\ \vdots \\ y_n
	\end{pmatrix} 
	\begin{pmatrix}
		x_1 & \dots & x_n
	\end{pmatrix}
	:= 
	\begin{pmatrix}
		x_1 y_1 & x_2 y_1 & \dots & x_n y_1 \\
		x_1 y_2 & x_2 y_2 & \dots & x_n y_2 \\
		\vdots & \vdots & \ddots & \vdots \\
		x_1 y_n & x_2 y_m & \dots & x_n y_m 
	\end{pmatrix} \)
\end{example}

\begin{example}
	\(
		\begin{pmatrix}
			1 & 0 & 2 \\
			0 & -1 & 3
		\end{pmatrix}
		\times
		\begin{pmatrix}
			2 & -1 \\
			0 & 5 \\
			1 & 1
		\end{pmatrix}
		=
		\begin{pmatrix}
			1 * 2 + 0 * 0 + 2 * 1 & 1 * (-1) + 0 * 5 + 2 * 1 \\
			0 * 2 + (-1) * 0 + 3 * 1 & 0 * (-1) + (-1) * 5 + 3 * 1
		\end{pmatrix}
		=
		\begin{pmatrix}
			4 & 1 \\
			3 & -2
		\end{pmatrix}
	\)
\end{example}

	\section{Лекция 2}

\subsection{Сумма}

$S_p, S_{p + 1}, \dots, S_q$ -- набор чисел

$ \sum_{i = p}^q S_i := S_p + S_{p + 1} + \dots + S_q $ -- сумма по $i$ от $p$ до $q$

$ \sum_{i=1}^100 i^2 = 1^2 + 2^2 + \dots + 100^2 $


Свойства
\begin{enumerate}
    \item $\lambda \sum_{i=1}^n S_i = \sum_{i=1}^q \lambda S_i $
    \item $\sum_{i=1}^q (S_i + t_i) = \sum_{i=1}^n S_i + \sum^n_{i=1} t_i $
    \item $\sum_{i=1}^m \sum_{j=1}^n S_{ij} = \sum_{j=1}^n \sum_{i=1}^m $ -- Сумма всех элементов матрицы $S = (s_{ij})$
\end{enumerate}

\subsection{Умножение матриц}

$A \in \text{Mat}_{m \times n}, B \in \text{Mat}_{n \times p} $

$AB = C$

$c_{ij} = A_{(i)} B^{(j)} = a_{i1} b_{1j} + a_{i2} b_{2j} + \dots + a_{1n} b_{nj} = \sum_{k=1}^n a_{ik} b_{kj} $

Свойства умножения матриц:
\begin{enumerate}
    % дописать размеры матриц сверху
    \item 
        $\underbracket{A(B + C)}_x = \underbracket{AB + AC}_y$ -- левая дистрибутивность
    
        Доказательство
        $ x_{ij} = A_{(i)} (B + C)^{(j)} = \sum_{k=1}^n a_{ik} (b_{kj} + c_{kj}) = \sum_{k=1}^n (a_{ik} b_{kj} + a_{ik} c_{kj}) = \sum_{k=1}^n a_{ik} b{kj} + \sum_{k=1}^n a_{ik} c_{kj} = A_{(i)} B^{(j)} + A_{(i)} C^{(j)} = y{ij} $
        % размеры матриц
        \item $(A+B)C = AC + BC$ -- правая дистрибутивность, доказывается аналогично
        \item $\lambda(AB) = (\lambda A) B = A (\lambda B)$
        % размеры матрицы нам ними
        \item $(AB)C = A(BC)$ -- ассоциативность

        Доказательство
        
        $\underbracket{(AB)C}_u = x, A\underbracket{(BC)}_v = y$

        \begin{align*}
            x_{ij} &= \sum_{k=1}^p u_{ik} * c_{kp} \\ 
            &= \sum_{k=1}^p (\sum_{l=1}^n a_{il} b_{lk}) c_{kj} \\ 
            &= \sum_{k=1}^p (\sum_{l=1}^n a_{il} b_{lk}) c_{kj}) \\ 
            &= \sum_{l=1}^n (\sum_{k=1}^p a_{il} b_{lk}) c{kj} \\ 
            &= \sum_{l=1}^n a_{il} (\sum_{k=1}^p b_{lk} c_{kj}) \\ 
            &= \sum_{l=1}^n a_{il} v_{lj} \\ 
            &= y_{ij}
        \end{align*}

    % размеры
    \item 
        $\underbracket{(AB)}_x^T = \underbracket{B^T A^T}_y$
        
        Доказательство

        \begin{align*}
            x_{ij} = [AB]_{ji} 
            = A_{(j)} B^{(i)} 
            = (B^T)_{(i)} (A^T)^{(j)}
            = y_{ij}
        \end{align*}
\end{enumerate}

Умножение матриц не коммутативно

$A = \begin{pmatrix} 0 & 1 \\ 0 & 0 \end{pmatrix}, 
B = \begin{pmatrix} 0 & 0 \\ 1 & 0 \end{pmatrix}$

$AB = \begin{pmatrix}1 & 0 \\ 0 & 0\end{pmatrix}, 
BA = \begin{pmatrix}0 & 0 \\ 0 & 1\end{pmatrix}$

\begin{definition}
$A \in \text{Mat}_{n \times n} \implies A$ называется \textit{квадртаной матрицей} подярка $n$
\end{definition}

Обозн.: $M_n := \text{Mat}_{n \times n}$

$A \in M_n$
% картинки главной и побочной диагоналей

\begin{definition}
    Матрица $A \in M_n$ называется \textit{диагональной} если все ее элементы вне главной диагонали равны нулю ($a_{ij} = 0$ при $i \neq j$)
\end{definition}

% квадратная матрица с болшими нулями вне диагонали
$A = \implies A = diag(a_1, a_2, \dots, a_n)$

\begin{lemma}
    $A = diag(a_1, \dots, a_n) \in M_n \implies$
    \begin{enumerate}
    \item $\forall B \in \text{Mat}_{n \times p}, AB = \begin{pmatrix}
            a_1 B_{(1)} \\
            a_2 B_{(2)} \\
            \vdots \\
            a_n B_{(n)} 
        \end{pmatrix}$
    \item $\forall B \in \text{Mat}_{m \times n}$ -- аналогично (вектор строка)
    \end{enumerate}
\end{lemma}

Доказательство
\begin{enumerate}
\item $[AB]_{ij} = \begin{pmatrix} 0 & \dots & 0 & a_i & 0 & \dots & 0\end{pmatrix} \begin{pmatrix} b_{1j} \\ b_{2j} \\ \vdots \\ b_{nj} \end{pmatrix} = a_i b_{ij} $
% поправить второй случай аналогично
\item $[BA]_{ij} = $ 
\end{enumerate}

% ЧТД (квадратик)

\begin{definition}
    % картинка с единицами на диагонали
    Матрица $E = E_n = diag(1, 1, \dots, 1)$ называется \textit{единичной матрицей} порядка $n$. 
\end{definition}

Свойства
\begin{enumerate}
    \item $EA = A \quad \forall A \in \text{Mat}_{n \times p}$
    \item $AE = A \quad \forall A \in \text{Mat}_{p \times n}$
    \item $AE = EA = A \quad \forall A \in M_n$
\end{enumerate}

\begin{definition}
    \textit{Следом} матрицы $A \in M_n$ называется число $trA = a_{11} + a_{22} + \dots + a_{nn} = \sum_{i=1}^n a_{ii}$ 
\end{definition}

Свойства
\begin{enumerate}
\item $tr(A + B) = trA + trB$
\item $tr(\lambda A) = \lambda tr A$
\item $tr(A^T) = tr(A)$
\item $tr(AB) = tr(BA) \forall A \in \text{Mat}_{m \times n}, B \in \text{Mat}_{nm}$

    Доказательство

    $AB = x \in M_m, BA = y \in M_n$

    $tr x = \sum_{i=1}^m x_{ii} = \sum_{i=1}^m \sum_{j=1}^n a_{ij} b_{ji}= \sum_{j=1}^n \sum_{i=1}^m b_{ji} a_{ij} = \sum_{j=1}^n y_{ij} = tr y$
\end{enumerate}

\begin{example}
    $A = (1, 2, 3), B = \begin{pmatrix}4 \\ 5 \\ 6\end{pmatrix}$

    $tr(AB) = tr(1 \cdot 4 + 2 \cdot 5 + 3 \cdot 6) = 32$

    $tr(BA) = \begin{pmatrix} 4 & 8 & 12 \\ 5 & 10 & 15 \\ 6 & 12 & 18 \end{pmatrix} = 4 + 10 + 18 = 32$
\end{example}


\subsection{Системы линейных уравнений}

\begin{definition}
    \textit{Линейное уравнение} -- $a_1 x_1 + a_2 x_2 + \dots + a_n x_n = b$

    $a_1, a_2, \dots, a_n, b$ -- коэффициенты

    $x_1, x_2, \dots, x_n$ -- неизвестные
\end{definition}

Система линейных уравнений

\begin{align*}
    a_{11} x_1 + a_{12} x_2 + \dots + a_{1n} x_n &= b_1 \\
    a_{21} x_1 + a_{22} x_2 + \dots + a_{2n} x_n &= b_2 \\
    \vdots & \\
    a_{m1} x_1 + a_{m2} x_2 + \dots + a_{mn} x_n &= b_m
\end{align*}

$a_{ij}, b_i \in \RR$

Основная задача: решить СЛУ

\begin{example}
    $n = m = 1$

    $ax = b$

    $a \neq 0 \implies x = \frac{b}{a}$

    $a = 0 \implies 0x = b$

    $b \neq 0 \implies$ нет решений

    $b = 0 \implies x \in \RR$ -- бесконечно много решений
\end{example}

\[
    A \in Mat_{m \times n}(R) = \begin{pmatrix}
        a_{11} & a_{12} & \dots & a_{1n} \\
        a_{21} & a_{22} & \dots & a_{2n} \\
        \vdots & \vdots & \ddots & \vdots \\
        a_{m1} & a_{m2} & \dots & a_{mn}
    \end{pmatrix} \text{ -- матрица коэффициентов}
\]

\[
    B \in \text{Mat}_{m \times 1} = \begin{pmatrix}
        b_1 \\ b_2 \\ \vdots \\ b_n
    \end{pmatrix} \text{ -- столбец правых частей}
\]

\[
    X \in \text{Mat}_{m \times 1} = \begin{pmatrix}
        x_1 \\ x_2 \\ \vdots \\ x_n
    \end{pmatrix} \text{ -- столбец неизвестных}
\]


$(*) \leftrightarrow Ax = b$ ---
Матричная форма записи СЛУ

\begin{definition}
    СЛУ называется 

    -- \textit{совместной}, если у нее есть хотя бы одно решение

    -- \textit{несовмествной}, если решений нет
\end{definition}


	\section{Семинар 1}

\subsection{Контакты}
Трушин Дмитрий Витальевич -- Дима

trushindima@yandex.ru

\subsection{Матрицы}
\subsubsection{Аномалии}

\begin{enumerate}
\item 
    $A \cdot B \neq B \cdot A$

    $\begin{pmatrix} 0 & 1 \\ 0 & 0 \end{pmatrix} 
    \begin{pmatrix} 0 & 0 \\ 1 & 0 \end{pmatrix}
    =
    \begin{pmatrix} 1 & 0 \\ 0 & 0 \end{pmatrix}$

    $\begin{pmatrix} 0 & 0 \\ 1 & 0 \end{pmatrix}
    \begin{pmatrix} 0 & 1 \\ 0 & 0 \end{pmatrix} 
    =
    \begin{pmatrix} 0 & 0 \\ 0 & 1 \end{pmatrix}$

    $\begin{pmatrix} 1 & 0 \\ 0 & 0 \end{pmatrix} 
    \begin{pmatrix} 0 & 0 \\ 0 & 1 \end{pmatrix} = 0$

    $A \neq 0 \cdot B \neq 0 = 0$

\item 
    $A \neq 0$
    
    $A^2 = 0$

    $\begin{pmatrix} 0 & 0 \\ 1 & 0 \end{pmatrix}^2 = 0$
    
    $\begin{pmatrix} 0 & 1 \\ 0 & 0 \end{pmatrix}^2 = 0$
\end{enumerate}

\subsubsection{Блочные операции}

\begin{tabular}{|c|c|}
    \hline
    A & B \\
    \hline
    C & D \\
    \hline
\end{tabular}

\begin{tabular}{|c|c|}
    \hline
    X & Y \\
    \hline
    Z & W \\
    \hline
\end{tabular}

\begin{tabular}{|c|c|}
    \hline
    $AX + BZ$ & $AY + BW$ \\
    \hline
    $CY + DZ$ & $CY + DW$ \\
    \hline
\end{tabular}


$A \cdot B = A \cdot (B_1, B_2, \dots, B_N) $

$B = (B_1, B_2, \dots, B_N) = (AB_1, AB_2, \dots, AB_n)$


$A = (A_1, \dots, A_n)$

$B = (B_1, \dots, B_n)$

$AB^T = (A_1, \dots, A_n) \begin{pmatrix} B_1^T \\ \vdots \\ B_N^T \end{pmatrix} = A_1 B_1^T + \dots + A_n B_n^T$

\subsubsection{Кек}

% матрица с единицами выше главной диагонали, остальное нули * A = матрица A сдвинутая вверх
% A * эту же матрицу --- ->
% Если же в этой матрице в последний ряд добавить единичку, тогда она при унможении будет циклически сдвигать строки
% А еще так можно можно делать любую перестановку строк и столбцов




\subsubsection{Лол}

$(A \cdot B)^T = B^T \cdot A^T$

\subsubsection{Хех}

$
\begin{tabular}{|c|c|}
    \hline
    A & B \\
    \hline
    C & D \\
    \hline
\end{tabular}^T
$

\begin{tabular}{|c|c|}
    \hline
    $A^T$ & $C^T$ \\
    \hline
    $B^T$ & $D^T$ \\
    \hline
\end{tabular}

\subsubsection{Мда}

% J_0 = матрица с нулями выше диагонали

$x \in M_n(\RR)$

$x J_0 = J_0 x$




	\section{Лекция 3}

$Ax = b, A \in \text{Mat}_{m \times n}, b \in \RR^m$

Полная информация о СЛУ содержится в её \textit{расширенной матрице} $(A | b)$.

\begin{definition}
    Две системы уравнений от одних и тех же неизвестных называются \textit{эквивалентными}, если они имеют одинаковые множества решений.
\end{definition}

\begin{example}
    Рассмотрим несколько СЛУ

    \begin{enumerate}[label=\Alph*)]
    \item
        $\begin{cases}
            x_1 + x_2 = 1 \\
            x_1 - x_2 = 0
        \end{cases}$

        $\begin{pmatrix}
            1 & 1 & 1 \\
            1 & -1 & 0
        \end{pmatrix}$

    \item
        $\begin{cases}
            2x_1 = 1 \\
            2x_2 = 1
        \end{cases}$

        $\begin{pmatrix}
            2 & 0 & 1 \\
            0 & 2 & 1
        \end{pmatrix}$

    \item
        $x_1 + x_2 = 1$
        
        $\begin{pmatrix} 1 & 1 & 1 \end{pmatrix}$
    \end{enumerate}

    A и B эквиваленты, так как обе имеют единственное решение $(\frac{1}{2}, \frac{1}{2})$.
    
    A и C не эквивалентны, так как C имеет бесконечно много решений.
\end{example}

\subsection{Как решить СЛУ?}

\textbf{Идея}: выполнить преобразование СЛУ, сохраняющее множество её решений, и привести её к такому виду, в котором СЛУ легко решается.

\begin{example}
    $\begin{pmatrix}
        1 & 0 & \dots & 0 \\
        0 & 1 & \dots & 0 \\
        \vdots & \vdots & \ddots & \vdots \\
        0 & 0 & \dots & 1
    \end{pmatrix} \leftrightarrow \begin{cases}
        x_1 = b_1 \\
        x_2 = b_2 \\ 
        \vdots \\ 
        x_n = b_n
    \end{cases}$
\end{example}

\subsubsection{Элементарные преобразования СЛУ и её расширенная матрица}

\begin{tabular}{c|c|c}
    тип & СЛУ & расширенная матрица \\
    \hline
    1. & K $i$-му уравнению прибавить $j$-ое, умноженное на $\lambda \in \RR \ (i \neq j)$ & $\text{Э}_1(i, j, \lambda)$ \\
    2. & Переставить $i$-е и $j$-е уравнения $(i \neq j)$ & $\text{Э}_2(i, j)$ \\
    3. & Умножить $i$-ое уравнение на $\lambda \neq 0$ & $\text{Э}_3(i, \lambda)$
\end{tabular}

\begin{enumerate}
\item
    $\text{Э}_1(i, j, \lambda):$ к $i$-ой строке прибавить $j$-ую, умноженную на $\lambda$ (покомпонентно), 

    $a_{ik} \to a_{ik} + \lambda a_{jk} \forall k = 1, \dots, n,$
    $b_i \to b_i + \lambda b_j$.
    
\item
    $\text{Э}_2(i, j):$ переставить i-ую и j-ую строки.

\item
    $\text{Э}_3(i, \lambda):$ умножить i-ю строку на $\lambda$ (покомпонентно).
\end{enumerate}

$\text{Э}_1, \text{Э}_2, \text{Э}_3$ называются \textit{элементарными преобразованиями строк расширенной матрицы}.

\begin{lemma}
    Элементарные преобразования СЛУ не меняют множество решений
\end{lemma}

\begin{proof}
    Пусть мы получили СЛУ$(\star\star)$ из СЛУ($\star$) путем элементарных преобразований.

    \begin{enumerate}
    \item
        Всякое решение системы $(\star)$ является решением $(\star\star)$.
    \item
        $(\star)$ получается из $(\star\star)$ путем элементарных преобразований.

        \begin{tabular}{c|c|}
            $(\star) \to (\star\star)$ & $(\star\star) \to (\star)$ \\
            \hline
            $\text{Э}_1(i, j, \lambda)$ & $\text{Э}_1(i, j, -\lambda)$ \\
            $\text{Э}_2(i, j)$ & $\text{Э}_2(i, j)$ \\
            $\text{Э}_3(i, \lambda)$ & $\text{Э}_3(i, \frac{1}{\lambda})$
        \end{tabular}
    \end{enumerate} 

    Следовательно, всякое решение ($\star\star$) является решением ($\star$) $\implies$ множества решений совпадают.
\end{proof}

\subsection{Ступенчатые матрицы}

\begin{definition}
    \textit{Ведущим элементом} ненулевой строки называется первый её ненулевой элемент.
\end{definition}

\begin{definition}
    Матрица $M \in \text{Mat}_{m \times n}$ называется \textit{ступенчатой}, или имеет ступенчатый вид, если:
    \begin{enumerate}
    \item Номера ведущих элементов её ненулевых строк строго возрастают.
    \item Все нулевые строки стоят в конце.
    \end{enumerate}
\end{definition}
\begin{equation*}
    M = \begin{pmatrix}
        0 & \dots & 0 & \diamond & * & * & * & * & * & * \\
        0 & \dots & 0 & 0 & \dots & \diamond & * & * & * & * \\
        0 & \dots & 0 & 0 & \dots & 0 & 0 & \diamond & * & * \\
        \vdots & \vdots & \vdots & \vdots & \vdots & \vdots & \vdots & \vdots & \vdots & \vdots \\
        0 & 0 & 0 & \dots & \dots & \dots & \dots & 0 & \diamond & * \\
        0 & 0 & 0 & \dots & 0 & 0 & 0 & 0 & 0 & 0
    \end{pmatrix}
.\end{equation*}
$\diamond \neq 0$, $*$ -- что угодно. 

\begin{definition}
    M имеет \textit{улучшенный ступенчатый вид}, если:

    \begin{enumerate}
    \item M имеет обычный ступенчатый вид.
    \item Все ведущие элементы равны 1.
    \item В одном столбце с любым ведущим элементом стоят только нули.
    \end{enumerate}
\end{definition}

\begin{equation*}
    M = \begin{pmatrix}
        0 & \dots & 0 & 1 & * & 0 & * & 0 & 0 & * \\
        0 & \dots & 0 & 0 & \dots & 1 & * & 0 & 0 & * \\
        0 & \dots & 0 & 0 & \dots & 0 & 0 & 1 & 0 & * \\
        \vdots & \vdots & \vdots & \vdots & \vdots & \vdots & \vdots & \vdots & \vdots & \vdots \\
        0 & 0 & 0 & \dots & \dots & \dots & \dots & 0 & 1 & * \\
        0 & 0 & 0 & \dots & 0 & 0 & 0 & 0 & 0 & 0
    \end{pmatrix}
.\end{equation*}

\begin{theorem}
    \begin{enumerate*}[label=\arabic*)]
    \item Всякую матрицу элементарными преобразованиями можно привести к ступенчатому виду.
    \item Всякую ступенчатую матрицу элементарными преобразованиями строк можно привести к улучшенному ступенчатому виду.
    \end{enumerate*}
\end{theorem}

\begin{consequence}
    Всякую матрицу элементарными преобразованиями строк можно привести к улучшенному ступенчатому виду.
\end{consequence}

\begin{proof}~
    \begin{enumerate}
    \item
        Алгоритм. Если M - нулевая, то конец. Иначе:
        \begin{enumerate}[label=Шаг \arabic*]
        \item Ищем первый ненулевой столбец, пусть $j$ --- его номер.
        \item Переставляем строки, если нужно, добиваемся того, что $a_{1j} \neq 0$
        \item 
            Выполняем $\text{Э}_1(2, 1, -\frac{a_{2j}}{a_{1j}}), \dots, \text{Э}_1(m, 1, -\frac{a_{mj}}{a_{1j}})$.
            В результате $a_{ij} = 0$ при $i = 2, 3, \dots m$.
        \end{enumerate}
        Дальше все повторяем для меньшей матрицы $M'$.
    \item 
        Алгоритм. Пусть $a_{1j_1}, a_{2j_2}, \dots, a_{rj_r}$ -- ведущие элементы ступенчатой матрицы.
        \begin{enumerate}[label=Шаг \arabic*]
        \item Выполняем $\text{Э}_3(1, \frac{1}{a_{1j_1}}), \dots, \text{Э}_3(r, \frac{1}{a_{rj_r}})$, в результате все ведущие элементы равны 1.
        \item Выполняем $\text{Э}_1(r - 1, r, -a_{r - 1 \ j_r}), \text{Э}_1(r - 2, r, -a_{r - 2 \ j_r}), \dots, \text{Э}_1(1, r, -a_{1 \ j_r})$. В результате все элементы над $a_{r j_r}$ равны 0.
        \end{enumerate}
        Аналогично обнуляем элементы над всеми остальными ведущими.
    \end{enumerate}
    Итог: матрица имеет улучшенный ступенчатый вид.
\end{proof}

\subsection{Применение элементарных преобразований СЛУ к матрицам}


Всякое элементарное преобразование строк матрицы реализуется умножением как умножение слева на подходящую ``элементарную матрицу''.

\begin{itemize}
\item 
    $\text{Э}_1(i, j, \lambda): A \to U_1(i, j, \lambda)A$, где
    \begin{equation*}
        U_1(i, j, \lambda) = \bordermatrix{
            &   &   &   &   &   & j &   \cr
            & 1 & 0 & 0 & \dots & 0 & 0 & 0 \cr
          i & 0 & 1 & 0 & \dots & 0 & \lambda & 0 \cr
            & 0 & 0 & 1 & \dots & 0 & 0 & 0 \cr
            & \vdots & \vdots & \vdots & \ddots & \vdots & \vdots & \vdots \cr
            & 0 & 0 & 0 & \dots & 1 & 0 & 0 \cr
            & 0 & 0 & 0 & \dots & 0 & 1 & 0 \cr
            & 0 & 0 & 0 & \dots & 0 & 0 & 1
        }
    \end{equation*}

    (на диагонали стоят единицы, на $i$-м $j$-м месте стоит $\lambda$, остальные элементы нули)

\item

    $\text{Э}_2(i, j): A \to U_2(i, j)A$, где
    \begin{equation*}
        U_2(i, j) = \bordermatrix{
            &   & i &   &   &   & j &   \cr
            & 1 & 0 & 0 & \dots & 0 & 0 & 0 \cr
          i & 0 & 0 & 0 & \dots & 0 & 1 & 0 \cr
            & 0 & 0 & 1 & \dots & 0 & 0 & 0 \cr
            & \vdots & \vdots & \vdots & \ddots & \vdots & \vdots & \vdots \cr
            & 0 & 0 & 0 & \dots & 1 & 0 & 0 \cr
          j & 0 & 1 & 0 & \dots & 0 & 0 & 0 \cr
            & 0 & 0 & 0 & \dots & 0 & 0 & 1
        }
    \end{equation*}

    (на диагонали стоят единицы, кроме $i$-го и $j$-го столбца (там нули, на $i$-м $j$-м и $j$-м $i$-м местах стоит 1, остальные нули)

\item 
    $\text{Э}_3(i, \lambda): A \to U_3(i, \lambda)A$, где
    \begin{equation*}
        U_3(i, \lambda) = \bordermatrix{    
            &   & i &   &   &   &   &   \cr
            & 1 & 0 & 0 & \dots & 0 & 0 & 0 \cr
          i & 0 & \lambda & 0 & \dots & 0 & 0 & 0 \cr
            & 0 & 0 & 1 & \dots & 0 & 0 & 0 \cr
            & \vdots & \vdots & \vdots & \ddots & \vdots & \vdots & \vdots \cr
            & 0 & 0 & 0 & \dots & 1 & 0 & 0 \cr
            & 0 & 0 & 0 & \dots & 0 & 1 & 0 \cr
            & 0 & 0 & 0 & \dots & 0 & 0 & 1
        }
    \end{equation*}

    (на диагонали стоят единицы, кроме i-го столбца, там $\lambda$, остальные элементы нули)
\end{itemize}

\begin{exercise}
    Доказательство.
\end{exercise}

\begin{exercise}
    Элементарные преобразования столбцов.
\end{exercise}


\end{document}
