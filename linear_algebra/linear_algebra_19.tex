\documentclass[a4paper]{article}

\usepackage[T2A]{fontenc}
\usepackage[utf8]{inputenc}
\usepackage[english, russian]{babel}

\usepackage[top=0.8in,bottom=0.8in,left=0.8in,right=0.8in,headheight=110pt]{geometry}
\usepackage{mathtools, amsthm, amssymb}
\usepackage{indentfirst}
\usepackage{enumitem}
\usepackage[colorlinks=true, linkcolor=blue]{hyperref}

\newcommand{\NN}{\mathbb{N}}
\newcommand{\ZZ}{\mathbb{Z}}
\newcommand{\QQ}{\mathbb{Q}}
\newcommand{\RR}{\mathbb{R}}

\theoremstyle{definition}
\newtheorem{definition}{Определение}
\newtheorem*{consequence}{Следствие}
\newtheorem*{example}{Пример}

\theoremstyle{remark}
\newtheorem*{exercise}{Упражнение}
\newtheorem*{remark}{Примечание}
\newtheorem*{answer}{Ответ}
\newtheorem*{hint}{Подсказка}

\theoremstyle{plain}
\newtheorem*{lemma}{Лемма}
\newtheorem{theorem}{Теорема}
\newtheorem{problem}{Задача}

\title{Линейная алгебра и геометрия}
\author{Slava Boben}
\date{September 9, 2019}

\begin{document}
	\pagestyle{empty}
	\maketitle
	\tableofcontents
	\newpage
	
	\pagestyle{plain}
	
	\section{Лекция 1}

\subsection{Общая информация}
\subsubsection{Контакты}
Авдеев Роман Сергеевич
\begin{itemize}[noitemsep]
	\item suselr@yandex.ru
	\item ravdeev@hse.ru
\end{itemize}

\subsubsection{О дисциплине}
1 -- 4 модули

Письменный экзамен: 2, 4 модули
\subsubsection{Оценка}
\begin{enumerate}[nosep]
	\item Экзамен
	\item Коллоквиум
	\item Контрольная работа
	\item Больше ДЗ
	\item Работа на семинарах
	\item Бонус -- Задачи из листков
\end{enumerate}

\begin{equation*}
	O_\text{Итог} = \min(10, \text{Округление}(0.4 * O_\text{Экз} + 0.22 * O_\text{Колл} + 0.16 * O_\text{КР} + 0.16 * O_\text{ДЗ} + 0.08 * O_\text{Сем} + 0.08 * O_\text{Л}), 10) \\
\end{equation*}
\begin{equation*}
	\text{Округление}(x) = [x]
\end{equation*}

\subsubsection{Содержание курса}

\begin{enumerate}
	\item Начало алгебры --- 9 -- 10 занятий
	
	\begin{itemize}[nosep]
		\item Матрицы
		\item Системы линейных уравнений
		\item Определители
		\item Комплексные числа
	\end{itemize}

	\item Собственно линейная алгебра
	\begin{itemize}
		\item Вектороное пространство
	\end{itemize}
\end{enumerate}

\subsection{Матрицы}
\begin{definition}
	Матрица размера \( n \times m \) --- это прямоугольная таблица высоты \( m \) и ширины \( n \)
\end{definition}
\[ 
	A = 
	\begin{pmatrix}
		a_{11}  & a_{12} & \dots & a_{1n} \\
		a_{21} & a_{22} & \dots & a_{2n} \\
		\vdots & \vdots & \ddots & \vdots \\
		a_{m1} & a_{m2} & \dots & a_{mn}
	\end{pmatrix}
\]

\(a_ij\) -- элемент на пересечении i-й строки и j-го столбца

Краткая запись -- \( A = (a_{ij}) \)

Множество всех матриц размера \(m \times n \) с коэффициентами из \( \RR \) (множество всех действительных чисел) --- \( \text{Mat}_{n*m}(\RR) \) или \( \text{Mat}_{n*m} \)

\begin{definition}
	Две матрицы \( A \in \text{Mat}_{n \times m} \) и \( B \in \text{Mat}_{p \times q} \) называются \textit{равными}, если \( m = p, n = q \), и соответствующие элементы равны
\end{definition}

\begin{example}
	\(
		\begin{pmatrix}
			\circ & \circ & \circ \\ \circ & \circ & \circ
		\end{pmatrix}
		\neq
		\begin{pmatrix}
			\circ & \circ \\ \circ & \circ \\ \circ & \circ
		\end{pmatrix}
	\)
\end{example}
% Пример с матрицами 2x3 и 3x2

\subsubsection{Операции над матрицами}

\( A, B \in \text{Mat}_{m*n} \)

\begin{itemize}
	\item \emph{Сумма} \( A + B := (a_{ij} + b_{ij}) \)
	\item \emph{Произведение на скаляр} \( \alpha \in \RR \implies \lambda A := (\lambda a_{ij})\)
\end{itemize}

Свойства суммы и произведения на скаляр

\( \forall A, B, C  \in \text{Mat}_{m*n} \forall \lambda, \mu \in \RR \)
\begin{enumerate}[label=(\arabic*), nosep]
	\item \( A + B = B + A \) (коммутативность)
	\item \( (A + B) + C = A + (B + C)\) (ассоциативность)
	\item \( A + 0 = 0 + A = A\), где \( 0 = \) % матрица нулей
	\item \( A + (-A) = 0 \)
	
	\( -A \) -- Противоположная матрица
	\item \( (\lambda + \mu) A = \lambda A + \mu A \)
	\item \( \lambda (A + B) = \lambda A + \lambda B \)
	\item \( \lambda (\mu A) = \lambda \mu A \)
	\item \( 1 A = A \)
\end{enumerate}

\begin{exercise}
	Доказать эти свойства
\end{exercise}

\begin{remark}
	Из свойств (1) -- (8) следует, что \( \text{Mat}_{n*m}(\RR) \) является векторным пространством над \( \RR \)
\end{remark}

\subsubsection{\(\RR^n\)}
\( \RR^n := \{ (x_1, \dots, x_n) \mid x_i \in \RR \ \forall i = 1, \dots, n \} \)

\( \RR^1 = \RR \) \quad -- числовая прямая

\( \RR^2 \) \quad -- плоскость

\( \RR^3 \) \quad -- трехмерное пространство \\

Договоримся отождествлять \( \RR^n \) со столбцами высоты \( n \)

\( (x_1, \dots, x_n) \leftrightarrow \left( \begin{smallmatrix}
	x_1 \\ \vdots \\ x_n
\end{smallmatrix} \right) \text{ --- ``вектор столбец''} \)

\( \RR^n \leftrightarrow \text{Mat}_{n * m}(\RR) \)

\( \left[ x = \begin{pmatrix}
	x_1 \\ \vdots \\ x_n
\end{pmatrix} \in \RR^n, y = \begin{pmatrix}
	y_1 \\ \vdots \\ y_n
\end{pmatrix} \in \RR^n \right] \implies \left[ x = y \iff x_i = y_i \forall i \right] \)

\( x + y := \begin{pmatrix}
	x_1 + y_1 \\ \vdots \\ x_n + y_n
\end{pmatrix} \)

\( \lambda \in \RR \implies \lambda x_i := (\lambda x_1, \dots) \)

\subsubsection{Транспонирование}

\( A \in \text{Mat}_{m*n} = \begin{pmatrix}
	a_{11} & a_{12} & \dots & a_{1n} \\
	a_{21} & a_{22} & \dots & a_{2n} \\
	\vdots & \vdots & \ddots & \vdots \\
	a_{m1} & a_{m2} & \dots & a_{mn}
\end{pmatrix} \leadsto A^T \in \text{Mat}_{n*m} := \begin{pmatrix}
a_{11} & a_{21} & \dots & a_{m1} \\
a_{12} & a_{22} & \dots & a_{m2} \\
\vdots & \vdots & \ddots & \vdots \\
a_{1n} & a_{2n} & \dots & a_{mn}
\end{pmatrix} \)

\( A^T\) ---  Транспонированная матрица

Свойства:
\begin{enumerate}[label=(\arabic*), nosep]
	\item \( (A^T)^T = A \)
	\item \( (A + B)^T = A^T + B^T \)
	\item \( (\lambda A)^T = \lambda A^T \)
\end{enumerate}

\begin{example}
	\( \begin{pmatrix}
		x_1 & \dots & x_n
	\end{pmatrix}^T = \begin{pmatrix}
		x_1 \\ \vdots \\ x_n
	\end{pmatrix} \)
\end{example}
\begin{example}
	\( \begin{pmatrix}
	x_1 \\ \vdots \\ x_n
	\end{pmatrix}^T = \begin{pmatrix}
	x_1 & \dots & x_n
	\end{pmatrix} \)
\end{example}
\begin{example}
	\( \begin{pmatrix}
		1 & 2 \\ 3 & 4 \\ 5 & 6
	\end{pmatrix}^T = \begin{pmatrix}
		1 & 3 & 5 \\
		2 & 4 & 6
	\end{pmatrix} \)
\end{example}

\subsubsection{Умножение матриц}

\( A = (a_{ij})\)

%добавить = колонка, = столбец

\( A_{(i)} \) -- \(i\)-я строка матрицы \( A \)

\( A^{(j)}\) -- \(j\)-й  столбец матрицы \( A \)

\begin{enumerate}[label=(\arabic*)]
	\item 
		Частный случай: Произведение строки на столбец одинаковой длинны 
	
		\( \underbrace{(x_1, \dots, x_n)}_{1 \times n} \underbrace{\begin{pmatrix}
			x_1 \\ \vdots \\ x_n
		\end{pmatrix}}_{n \times 1} = x_1 * y_1 + \dots + x_n * y_n \)
		
		%Подписи 1*n и n * 1
	\item
		\( A \) - матрица размера \( m * n \)
		
		\( B \) - матрица размера \( n * p \)
		
		Кол-во строк матрицы \( A \) равно кол-ву столбцов матрицы \( B \) --- условие согласованности матриц
		
		\( AB := C \in \text{Mat}_{m*p}\), где \( C_{ij} = A_{(i)} B^{(j)}\)
\end{enumerate}

\begin{example}
	\( \begin{pmatrix}
		y_1 \\ \vdots \\ y_n
	\end{pmatrix} 
	\begin{pmatrix}
		x_1 & \dots & x_n
	\end{pmatrix}
	:= 
	\begin{pmatrix}
		x_1 y_1 & x_2 y_1 & \dots & x_n y_1 \\
		x_1 y_2 & x_2 y_2 & \dots & x_n y_2 \\
		\vdots & \vdots & \ddots & \vdots \\
		x_1 y_n & x_2 y_m & \dots & x_n y_m 
	\end{pmatrix} \)
\end{example}

\begin{example}
	\(
		\begin{pmatrix}
			1 & 0 & 2 \\
			0 & -1 & 3
		\end{pmatrix}
		\times
		\begin{pmatrix}
			2 & -1 \\
			0 & 5 \\
			1 & 1
		\end{pmatrix}
		=
		\begin{pmatrix}
			1 * 2 + 0 * 0 + 2 * 1 & 1 * (-1) + 0 * 5 + 2 * 1 \\
			0 * 2 + (-1) * 0 + 3 * 1 & 0 * (-1) + (-1) * 5 + 3 * 1
		\end{pmatrix}
		=
		\begin{pmatrix}
			4 & 1 \\
			3 & -2
		\end{pmatrix}
	\)
\end{example}

\end{document}